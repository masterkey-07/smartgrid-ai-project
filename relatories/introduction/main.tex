\documentclass[a4paper,12pt]{article}
\usepackage[utf8]{inputenc}
\usepackage[T1]{fontenc}
\usepackage[brazilian]{babel}
\usepackage{indentfirst}
\usepackage{abntex2cite}
%\usepackage{hyperref}
\usepackage{graphicx}
\usepackage{amsmath}
\usepackage{setspace}
\usepackage{times}

\title{Título do Trabalho}
\author{João Vínicius\\
Victor Jorge Carvalho Chaves}
\date{11/07/2024}

\begin{document}

\maketitle
\begin{abstract}
    Este é o resumo do trabalho, onde serão descritos de forma sucinta os principais pontos abordados.
\end{abstract}

\tableofcontents
\newpage

\section{Introdução e Motivação}

Redes elétricas são responsáveis por realizar a geração,
transmissão e distribuição de energia em um território
e são fundamentais para o funcionamento da sociedade.

E conforme o passar dos anos, com o crescimento da sociedade,
há o aumento no consumo de energia elétrica.
Além disso, com as questões climáticas em jogo e a busca por mais fontes de energia limpa,
há a entrada de novos elementos nas redes elétricas, como painéis solares, aerogeradores, etc. Que aumentam a complexidade das redes.

E por fim, ocorreu vários casos no mundo de blackouts, que foram causados por mal funcionamentos da rede, ataques cibernéticos, falta de manutenção, etc.

E com crescimento das redes elétricas para atender a situação do mundo, emergiu o conceito de Smart Grid (Rede Elétrica Inteligente), redes elétricas que implementam múltiplas tecnologias para lidar com os desafios citados acima.

E dentre umas das tecnologias aplicadas em Smart Grids, é a inteligencia artificial, que pode resolver desafios de forecasting, detecção de ataques, e problemas de otimização.

\section{Conceitos Fundamentais}
\subsection{Smart Grid}
Sistema de energia elétrica que se utiliza da tecnologia da informação para fazer com que o sistema seja mais eficiente (econômica e energeticamente), confiável e sustentável.

A definição de redes elétricas inteligentes ainda não está completamente consolidada, mas nesse sistema devem constar os seguintes atributos

\begin{enumerate}
    \item Sistemas de transmissão e distribuição transparentes e controláveis;
    \item Fontes de energia renovável, geração distribuída e armazenamento de energia nos dois lados do medidor;
    \item Capacidade para resposta à demanda e controle de demanda.
\end{enumerate}

\section{Trabalhos Relacionados}
\subsection{Harmonized and Open Energy Dataset for Modeling a Highly Renewable Brazilian Power System}

Nesse trabalho é desenvolvido um conjunto de dados aberto para análise de cenários com modelos como o PyPSA.
Esse conjunto inclui dados de séries temporais, dados geoespaciais e dados tabulares sobre usinas e demandas de energia.
Isso facilita estudos adicionais focados na descarbonização do sistema energético brasileiro, mas pode ser auxiliar para outros estudo também.

\section{Objetivo}
De forma direta e sucinta, um parágrafo que resuma o que será feito neste trabalho.
Descreva claramente o objetivo principal do trabalho, destacando o que será alcançado.

\section{Metodologia Experimental}
Quais serão os passos e técnicas/biblioteca/tecnologias em geral que serão utilizadas para que seu projeto se concretize?
Detalhe a abordagem metodológica, os passos experimentais, as ferramentas e tecnologias que serão empregadas no desenvolvimento do projeto.

\section{O que será entregue no final?}
Esta parte é a mais importante, pois será a sua promessa de projeto e portanto, ela quem guiará sua nota final.
Explique quais serão os resultados finais, entregáveis ou produtos do seu trabalho, e como eles serão apresentados.

\section{Referências Bibliográficas}
\begin{itemize}
    \item Deng, Y., Cao, KK., Hu, W. et al. Harmonized and Open Energy Dataset for Modeling a Highly Renewable Brazilian Power System. Sci Data 10, 103 (2023). https://doi.org/10.1038/s41597-023-01992-9
    \item T. Brown, J. Hörsch, D. Schlachtberger, PyPSA: Python for Power System Analysis, 2018, Journal of Open Research Software, 6(1), arXiv:1707.09913, DOI:10.5334/jors.188
    \item SAP Insights. "The Smart Grid: How AI is Powering Today’s Energy Technologies." Disponível em: SAP Insights. Acesso em: 11 jul. 2024.
\end{itemize}
++
\end{document}
